\def\notes{1}
\documentclass[11pt]{article}
%\usepackage[T1]{fontenc}
%\usepackage[letterpaper]{geometry}
%\geometry{verbose,tmargin=3cm,bmargin=3cm,lmargin=3cm,rmargin=3cm}
%\usepackage{verbatim}
\usepackage{amsfonts,amsmath,amssymb,amsthm}
%\usepackage{setspace}
\usepackage{xspace}%,enumitem}
%\usepackage{times}
\usepackage{fullpage,array}
%\usepackage{hyperref}
%\doublespacing
\usepackage{color}
\usepackage{numdef}
\usepackage{enumitem}
\usepackage{amsopn}
%\usepackage{hyperref} 
\usepackage{mathrsfs}
\usepackage{float}
\restylefloat{table}
\usepackage{makecell}
\usepackage[linesnumbered,ruled]{algorithm2e}


\renewcommand\theadalign{bc}
\renewcommand\theadfont{\bfseries}
\renewcommand\theadgape{\Gape[4pt]}
\renewcommand\cellgape{\Gape[4pt]}
\providecommand{\sqbinom}{\genfrac{\lbrack}{\rbrack}{0pt}{}}
\DeclareMathOperator{\spn}{span}

\newif\ifdraft
%\drafttrue
\draftfalse
%%%%%%%%%%%%%%%%%%%%%%%%%%%%% Textclass specific LaTeX commands.
%\numberwithin{equation}{section} %% Comment out for sequentially-numbered
\numberwithin{figure}{section} %% Comment out for sequentially-numbered
\newtheorem{thm}{Theorem}[section]
\newtheorem{conjecture}[thm]{Conjecture}
\newtheorem{definition}[thm]{Definition}
\newtheorem{dfn}[thm]{Definition}
\newtheorem{lemma}[thm]{Lemma}
\newtheorem{remark}[thm]{Remark}
\newtheorem{proposition}[thm]{Proposition}
\newtheorem{corollary}[thm]{Corollary}
\newtheorem{claim}[thm]{Claim}
\newtheorem{fact}[thm]{Fact}
\newtheorem{openprob}[thm]{Open Problem}
\newtheorem{remk}[thm]{Remark}
\newtheorem{example}[thm]{Example}
\newtheorem{apdxlemma}{Lemma}
\newcommand{\question}[1]{{\sf [#1]\marginpar{?}} }
\newcommand{\Mod}[1]{\ (\mathrm{mod}\ #1)}

%\usepackage{babel}
\usepackage{tikz}

\makeatletter
\newcommand*{\circled}{\@ifstar\circledstar\circlednostar}
\makeatother

\newcommand*\circledstar[1]{%
  \tikz[baseline=(C.base)]
    \node[%
      fill,
      circle,
      minimum size=1.35em,
      text=white,
      font=\sffamily,
      inner sep=0.5pt
    ](C) {#1};%
}
\newcommand*\circlednostar[1]{%
  \tikz[baseline=(C.base)]
    \node[%
      draw,
      circle,
      minimum size=1.35em,
      font=\sffamily,
      inner sep=0.5pt
    ](C) {#1};%
}

\providecommand{\bd}[1]{\circled{#1}}


%Eli's macros
\newcommand{\E}{{\mathbb{E}}}
%\newcommand{\N}{{\mathbb{N}}}
\newcommand{\R}{{\mathbb{R}}}
\newcommand{\calF}{{\cal{F}}}
%\newcommand{\inp}[1]{\langle{#1}\rangle}
\newcommand{\disp}[1]{D_{#1}}
%\newcommand{\spn}[1]{{\rm{span}}\left(#1\right)}
%\newcommand{\supp}{{\rm{supp}}}
\newcommand{\agree}{{\sf{agree}}}
\newcommand{\RS}{{\sf{RS}}}
\newcommand{\aRS}{{\sf{aRS}}}
\newcommand{\rrs}{R_{\aRS}}
\newcommand{\VRS}{V_\RS}
\newcommand{\x}{\ensuremath{\mathbf{x}}\xspace}
\renewcommand{\max}{\ensuremath{\mathrm{\mathbf{max}}}\xspace}
\newcommand{\summ}{\ensuremath{\sum_{i\in H}}\xspace}
\newcommand{\summa}{\ensuremath{\sum_{a\in H}}\xspace}
\renewcommand{\a}{\ensuremath{\mathbf{a}}\xspace}

\newcommand{\X}{\ensuremath{\mathbf{X}}\xspace}
\newcommand{\Y}{\ensuremath{\mathbf{Y}}\xspace}
\newcommand{\y}{\ensuremath{\mathbf{y}}\xspace}

\newcommand{\set}[1]{\ensuremath{\left\{#1\right\}}\xspace}
\newcommand{\angles}[1]{\langle{#1}\rangle}
%\renewcommand{\span}{\ensuremath{\mathsf{span}}\xspace}
\newcommand{\condset}[2]{\set{#1 \mid #2 }}
\newcommand{\zo}{\set{0,1}}
\newcommand{\zon}{{\zo^n}}
\newcommand{\zom}{{\zo^m}}
\newcommand{\zok}{{\zo^}}
\newcommand{\eps}{\epsilon}
\newcommand{\e}{\eps}
%\newcommand{\d}{\delta}
\newcommand{\poly}{\ensuremath{\mathrm{poly}(\lambda)}\xspace}
\newcommand{\polylog}{\ensuremath{\mathrm{polylog}(\lambda)}\xspace}
\newcommand{\itone}{{\it{(i)}}\xspace}
\newcommand{\ittwo}{{\it{(ii)}}\xspace}
\newcommand{\itthree}{{\it{(iii)}}\xspace}
\newcommand{\itfour}{{\it{(iv)}}\xspace}
\newcommand{\cc}{{\rm{CC}}}
\newcommand{\rnk}{{\rm{rank}}}
\newcommand{\T}{T}
\newcommand{\I}{{\mathbb{I}}}
\ifdraft
\newcommand{\ariel}[1]{{\color{blue}{\textit{#1 --- ariel gabizon}}}}
\else
\newcommand{\ariel}[1]{}
\fi
\providecommand{\improvement}[1]{{\color{red} \textbf{#1}}}
 \mathchardef\mhyphen="2D

\num\newcommand{\G1}{\ensuremath{{\mathbb G}_1}\xspace}
%\newcommand{\G}{\ensuremath{{\mathbb G}}\xspace}
\newcommand{\Gstar}{\ensuremath{{\mathbb G}^*}\xspace}

\num\newcommand{\G2}{\ensuremath{{\mathbb G}_2}\xspace}
%\num\newcommand{\G11}{\ensuremath{\G1\setminus \set{0} }\xspace}
%\num\newcommand{\G21}{\ensuremath{\G2\setminus \set{0} }\xspace}
\newcommand{\grouppair}{\ensuremath{G^*}\xspace}

\newcommand{\Gt}{\ensuremath{{\mathbb G}_t}\xspace}
\newcommand{\F}{\ensuremath{\mathbb F}\xspace}
\newcommand{\Fstar}{\ensuremath{\mathbb F^*}\xspace}

\newcommand{\help}[1]{$#1$-helper\xspace}
\newcommand{\randompair}[1]{\ensuremath{\mathsf{randomPair}(#1)}\xspace}
\newcommand{\pair}[1]{$#1$-pair\xspace}
\newcommand{\pairs}[1]{$#1$-pairs\xspace}

\newcommand{\pairone}[1]{\G1-$#1$-pair\xspace}
\newcommand{\pairtwo}[1]{\G2-$#1$-pair\xspace}
\newcommand{\sameratio}[2]{\ensuremath{\mathsf{SameRatio}(#1,#2)}\xspace}
\newcommand{\vecc}[2]{\ensuremath{(#1)_{#2}}\xspace}
\newcommand{\players}{\ensuremath{[n]}\xspace}
\newcommand{\adv}{\ensuremath{\mathcal A}\xspace}
\newcommand{\advprime}{\ensuremath{{\mathcal A}'}\xspace}
\newcommand{\extprime}{\ensuremath{E'}\xspace}
\newcommand{\advrand}{\ensuremath{\mathsf{rand}_{\adv}}\xspace}

\newcommand{\ci}{\ensuremath{\mathrm{CI}}\xspace}
\renewcommand{\deg}{\ensuremath{\mathrm{deg}}\xspace}
\newcommand{\pairvec}[1]{$#1$-vector\xspace}
\newcommand{\Fq}{\ensuremath{\mathbb{F}_q}\xspace}
\newcommand{\Fp}{\ensuremath{\mathbb{F}_p}\xspace}
\newcommand{\Fptwo}{\ensuremath{\mathbb{F}_{p^2}}\xspace}
\newcommand{\Fr}{\ensuremath{\mathbb{F}_r}\xspace}
\newcommand{\Ft}{\ensuremath{\mathbb{F}_t}\xspace}
\newcommand{\randpair}[1]{\ensuremath{\mathsf{rp}_{#1}}\xspace}
\newcommand{\randpairone}[1]{\ensuremath{\mathsf{rp}_{#1}^{1}}\xspace}
\newcommand{\abase}{\ensuremath{A_{\mathrm{\mathbf{0}}}}\xspace}
\newcommand{\bbase}{\ensuremath{B_{\mathrm{\mathbf{0}}}}\xspace}
\newcommand{\cbase}{\ensuremath{C_{\mathrm{\mathbf{0}}}}\xspace}

\newcommand{\amid}{\ensuremath{A_{\mathrm{mid}}}\xspace}
\newcommand{\bmid}{\ensuremath{B_{\mathrm{mid}}}\xspace}
\newcommand{\cmid}{\ensuremath{C_{\mathrm{mid}}}\xspace}

\newcommand{\negl}{\ensuremath{\mathsf{negl}(\lambda)}\xspace}
\newcommand{\randpairtwo}[1]{\ensuremath{\mathsf{rp_{#1}^2}}\xspace}%the randpair in G2
\newcommand{\nilp}{\ensuremath{\mathscr N}\xspace}
\newcommand{\snark}{\ensuremath{\mathscr S}\xspace}
\newcommand{\groupgen}{\ensuremath{\mathscr G}\xspace}
\newcommand{\qap}{\ensuremath{\mathscr Q}\xspace}

\newcommand{\rej}{\ensuremath{\mathsf{rej}}\xspace}
\newcommand{\acc}{\ensuremath{\mathsf{acc}}\xspace}
\newcommand{\res}{\ensuremath{\mathsf{res}}\xspace}
\newcommand{\sha}[1]{\ensuremath{\mathsf{COMMIT}(#1)}\xspace}
 \newcommand{\shaa}{\ensuremath{\mathsf{COMMIT}}\xspace}
 \newcommand{\comm}[1]{\ensuremath{\mathsf{comm}_{#1}}\xspace}
 \newcommand{\defeq}{:=}

\newcommand{\A}{\ensuremath{\vec{A}}\xspace}
\newcommand{\B}{\ensuremath{\set{0,1}}\xspace}
\newcommand{\C}{\ensuremath{\vec{C}}\xspace}
\newcommand{\Btwo}{\ensuremath{\vec{B_2}}\xspace}
\newcommand{\treevecsimp}{\ensuremath{(\tau,\rho_A,\rho_A \rho_B,\rho_A\alpha_A,\rho_A\rho_B\alpha_B, \rho_A\rho_B\alpha_C,\beta,\beta\gamma)}\xspace}% The sets of elements used in simplifed relation tree in main text body
\newcommand{\rcptc}{random-coefficient subprotocol\xspace}
\newcommand{\rcptcparams}[2]{\ensuremath{\mathrm{RCPC}(#1,#2)}\xspace}
\newcommand{\verifyrcptcparams}[2]{\ensuremath{\mathrm{\mathsf{verify}RCPC}(#1,#2)}\xspace}
\newcommand{\randadv}{\ensuremath{\mathsf{rand}_{\adv}}\xspace}
 \num\newcommand{\ex1}[1]{\ensuremath{ #1\cdot g_1}\xspace}
 \num\newcommand{\ex2}[1]{\ensuremath{#1\cdot g_2}\xspace}
 \newcommand{\pr}{\mathrm{Pr}}
 \newcommand{\powervec}[2]{\ensuremath{(1,#1,#1^{2},\ldots,#1^{#2})}\xspace}
\num\newcommand{\out1}[1]{\ensuremath{\ex1{\powervec{#1}{d}}}\xspace}
\num\newcommand{\out2}[1]{\ensuremath{\ex2{\powervec{#1}{d}}}\xspace}
 \newcommand{\nizk}[2]{\ensuremath{\mathrm{NIZK}(#1,#2)}\xspace}% #2 is the hash concatenation input
 \newcommand{\verifynizk}[3]{\ensuremath{\mathrm{VERIFY\mhyphen NIZK}(#1,#2,#3)}\xspace}
\newcommand{\protver}{protocol verifier\xspace} 
\newcommand{\mulgroup}{\ensuremath{\F^*}\xspace}
\newcommand{\lag}[1]{\ensuremath{L_{#1}}\xspace} 
\newcommand{\sett}[2]{\ensuremath{\set{#1}_{#2}}\xspace}
\newcommand{\omegaprod}{\ensuremath{\alpha_{\omega}}\xspace}
\newcommand{\lagvec}[1]{\ensuremath{\mathrm{LAG}_{#1}}\xspace}
\newcommand{\trapdoor}{\ensuremath{r}}
\newcommand{\trapdoorext}{\ensuremath{r_{\mathrm{ext}}}\xspace}
\newcommand{\trapdoorsim}{\ensuremath{r_{\mathrm{sim}}}\xspace}
\newcommand{\sk}{\ensuremath{\mathsf{sk}}\xspace}
\newcommand{\msg}{\ensuremath{\mathsf{m}}\xspace}
\newcommand{\pk}{\ensuremath{\mathsf{pk}}\xspace}
\newcommand{\pkagg}{\ensuremath{\mathsf{pk_{agg}}}\xspace}
\newcommand{\skagg}{\ensuremath{\mathsf{sk_{agg}}}\xspace}
\newcommand{\sign}{\ensuremath{\mathbf{Sign}}\xspace}
\newcommand{\phash}{\ensuremath{\mathbf{PedersenHash}}\xspace}
\newcommand{\ulphash}{\ensuremath{\mathbf{ULPedersenHash}}\xspace}
\newcommand{\bhash}{\ensuremath{\mathbf{Blake2s}}\xspace}
\newcommand{\ulpserialize}{\ensuremath{\mathbf{ULPSerialize}}\xspace}
\newcommand{\ulfieldhash}{\ensuremath{\mathbf{ULFieldHash}}\xspace}

\num\newcommand{\enc1}[1]{\ensuremath{\left[#1\right]_1}\xspace}
\num\newcommand{\enc2}[1]{\ensuremath{\left[#1\right]_2}\xspace}
\newcommand{\gen}{\ensuremath{\mathsf{Gen}}\xspace}
\newcommand{\prv}{\ensuremath{\mathsf{\mathbf{P}}}\xspace}
\newcommand{\prvpoly}{\ensuremath{\mathrm{P_{\mathsf{poly}}}}\xspace}
\newcommand{\verpoly}{\ensuremath{\mathrm{V_{\mathsf{poly}}}}\xspace}
\newcommand{\prf}{\ensuremath{\pi}\xspace}
\newcommand{\simprv}{\ensuremath{\mathrm{P^{sim}}}\xspace}

%\newcommand{\enc}[1]{\ensuremath{\left[#1\right]}\xspace}
%\num\newcommand{\G0}{\ensuremath{\mathbf{G}}\xspace}
\newcommand{\GG}{\ensuremath{\mathbf{G^*}}\xspace}  % would have liked to call this G01 but problem with name
\num\newcommand{\g0}{\ensuremath{\mathbf{g}}\xspace}
\newcommand{\inst}{\ensuremath{\phi}\xspace}
\newcommand{\inp}{\ensuremath{x}\xspace}
\newcommand{\wit}{\ensuremath{\omega}\xspace}
\newcommand{\ver}{\ensuremath{\mathsf{\mathbf{V}}}\xspace}
\newcommand{\per}{\ensuremath{\mathsf{\mathbf{P}}}\xspace}
\newcommand{\sonic}{\ensuremath{\mathsf{Sonic}}\xspace}
\newcommand{\aurora}{\ensuremath{\mathsf{Aurora}}\xspace}
\newcommand{\rel}{\ensuremath{\mathcal{R}}\xspace}
\newcommand{\ext}{\ensuremath{E}\xspace}
\newcommand{\params}{\ensuremath{\mathsf{params}_{\inst}}\xspace}
\newcommand{\protparams}{\ensuremath{\mathsf{params}_{\inst}^\advv}\xspace}
\num\newcommand{\p1}{\ensuremath{P_1}\xspace}
\newcommand{\advv}{\ensuremath{ {\mathcal A}^{\mathbf{*}}}\xspace} % the adversary that uses protocol adversary as black box
\newcommand{\crs}{\ensuremath{\sigma}\xspace}
%\num\newcommand{\crs1}{\ensuremath{\mathrm{\sigma}_1}\xspace}
%\num\newcommand{\crs2}{\ensuremath{\mathrm{\sigma}_2}\xspace}

\renewcommand{\sim}{\ensuremath{\mathsf{sim}}\xspace}%the distribution of messages when \advv simulates message of \p1
\newcommand{\real}{\ensuremath{\mathsf{real}}\xspace}%the distribution of messages when \p1 is honest and \adv controls rest of players
 \newcommand{\koevec}[2]{\ensuremath{(1,#1,\ldots,#1^{#2},\alpha,\alpha #1,\ldots,\alpha #1^{#2})}\xspace}
\newcommand{\mida}{\ensuremath{A_{\mathrm{mid}}}\xspace}
\newcommand{\midb}{\ensuremath{B_{\mathrm{mid}}}\xspace}
\newcommand{\midc}{\ensuremath{C_{\mathrm{mid}}}\xspace}
\newcommand{\chal}{\ensuremath{\mathsf{challenge}}\xspace}
\newcommand{\attackparams}{\ensuremath{\mathsf{params^{pin}}}\xspace}
\newcommand{\attackdist}[2]{\ensuremath{AD_{#1}}\xspace}
\renewcommand{\neg}{\ensuremath{\mathsf{negl}(\lambda)}\xspace}
\newcommand{\ro}{\ensuremath{{\mathcal R}}\xspace}
\newcommand{\elements}[1]{\ensuremath{\mathsf{elements}_{#1}}\xspace}
 \num\newcommand{\elmpowers1}[1]{\ensuremath{\mathrm{\mathsf{e}}^1_{#1}}\xspace}
\newcommand{\bls}{\ensuremath{\mathsf{BLS}}\xspace}
\newcommand{\elempowrs}[1]{\ensuremath{\mathsf{e}_{#1}}\xspace}
 \newcommand{\secrets}{\ensuremath{\mathsf{secrets}}\xspace}
 \newcommand{\polysofdeg}[1]{\ensuremath{\F_{< #1}[X]}\xspace}
 \newcommand{\bivar}[1]{\ensuremath{\F_{< #1}[X,Y]}\xspace}
 \newcommand{\sig}{\ensuremath{\sigma}\xspace}
 \newcommand{\prot}{\ensuremath{\mathscr{P}}\xspace}
 \newcommand{\sigprv}{\ensuremath{\mathsf{P_{sc}}}\xspace}
 \newcommand{\sigver}{\ensuremath{\mathsf{V_{sc}}}\xspace}
\newcommand{\bctv}{\ensuremath{\mathsf{BCTV}}\xspace}
\newcommand{\PI}{\ensuremath{\mathsf{PI}}\xspace}
\newcommand{\PIb}{\ensuremath{\mathsf{PI_B}}\xspace}
\newcommand{\PIc}{\ensuremath{\mathsf{PI_C}}\xspace}
\newcommand{\dl}[1]{\ensuremath{\widehat{#1}}\xspace}
%\newcommand{\sett}[2]{\ensuremath{\set{#1}_{#2}}\xspace}
\newcommand{\edcp}{\ensuremath{{E}_\mathtt{Ed/CP}}\xspace}
\newcommand{\gedcp}{\ensuremath{G_{{E}_\mathtt{Ed/CP}}}\xspace}

\title{%
Celo Light Client}
\date{\today}
%\author{}
\date{}
\begin{document}
\maketitle


\section{Introduction}
We assume we are given groups of prime order $r$ $\G1,\G2,\Gt$,
and denote by \F the finite field of the same size.
We are also given generators $g_1\in \G1, g_2\in \G2$.
We write \G1 and \G2 additively, and \Gt multiplicatively.

We denote by $H$ a hash function taking as input strings of arbitrary length
and outputting elements of $\G2$.
We model $H$ as a random oracle in the security proof, mainly as that is needed for security of the \bls signature scheme.
\paragraph{BLS signature scheme:}
\begin{itemize}
\item The message \msg is an arbitrary string 
\item The secret key \sk is a uniformly chosen element of \F, and the corresponding public key is $\pk \defeq \sk\cdot g_1$.
\item The signature \sig of \msg under \sk is 
\[\sig = \sign(\msg,\sk) \defeq  \sk\cdot H(\msg)\]
\end{itemize}

\bls has the extremely useful property that $\sign(\msg,\sk_1+\ldots \sk_t) = 
\sum_{i\in [t]} \sign(\msg,\sk_i)$

Thus, a set of users with private keys \sett{\sk_i}{i\in [t]}
and public keys \sett{\pk_i}{i\in [t]}
can sign \msg separately and their signatures can be aggregated to a single signature under public key $\pkagg \defeq \sum_{i\in [t]} \pk_i$.

\paragraph{Registered key owners:}

To avoid the so-called ``rogue key-attack'' when aggregating \bls signatures, we must only allow signatures with public keys $\pk_i$ such that a proof of knowledge of \sk has been provided.
Thus, when a key \pk is authorized to partcipate in commitees, such a zk proof of knowledge of \sk, e.g. Schnorr, must be provided.


\section{The light client protocol}
We refer to stake holders by their registered public key.
Thus, when we refer to a \emph{commitee}, we mean a set of public keys \set{\pk}.




Denote by $C_i$ the commitee of the $i$'th epoch.\footnote{We are assuming here an idealized consensus functionality where this value is well-defined; e.g. not dealing with forks when describing the light client.}
The light client \ver will only verify the identity of the current epoch commitee.
It will satisfy the following completeness and soundness properties

\paragraph{Completeness (liveness):}
If all commitees up to epoch $T$ have had a 2/3-honest majority, then
\ver will obtain the correct value $C_T$
\paragraph{Soundness:} If all commitees $\sett{C_i}{\in [T]}$ have more than $1/3$ honest players then \ver will not be convinced of a wrong value $C_T$.

\paragraph{Last block header of epoch}
The structure of the $n$'th block header of an epoch is important for the light client protocol;
and so we describe some of its details.
\begin{enumerate}
 \item It will contain a string $\msg^* = (S_1,S_2)$, for the two sets $S_1=\sett{\pk_i}{i\in [s]},S_2=\sett{\pk'_i}{i\in [s]}$ of the keys we add and remove from the validator set, i.e. $C_{T} = (C_{T-1}\setminus S_1)\cup S_2$.
 \item It will contain a string $x\in \B^t$ signifying what validators from $C_{T-1}$ signed $\msg\defeq (n,\msg^*)$.
 \item Let $\pkagg \defeq \sum_{i\in [t]} \pk_i$, where $C_{T-1} = \sett{pk_i}{\in [t]}$.
 The header contains the signature $\sig = \sign(\msg,\skagg)$.
 \end{enumerate}

\paragraph{Light client verification}
\ver receives, for $i\in [T-1]$
\begin{enumerate}
 \item A message $\msg_i =(S_{i,1},S_{i,2})$.
 \item A string $x_i\in \B^t$
 \item A signature $\sig_i$.
\end{enumerate}
\ver starts with the validator set $C_1$ which we assume is hard coded in the genesis block and agreed upon.

For each $i\in \set{2,\ldots,T}$ \ver 
\begin{enumerate}
 \item Checks that $x_i$ has at least $2/3\cdot t$ set bits, and sets $D_i\subset [t]$ to be the indices of the set bits of $x_i$.
 \item Computes $\pk\defeq \sum_{j\in D_i} \pk_j$.
% \item Checks that $\versig(\msg_i,\pk,\sig_i)$
 \item computes $C_{i+1} = (C_{i}\setminus S_{i,1})\cup S_{i,2}$.
 
\end{enumerate}

\section{Hash to group}

The BLS signature scheme uses $H: \{0,1\}^* \rightarrow \G2$ as a hash function that outputs random elements in $\G2$. It is modeled as a random oracle.

It is instantiated using a composition of a Pedersen hash defined over the curve $\edcp$ from \cite{bowe2018zexe}, and a few Blake2 hashes. 

\paragraph{Pedersen hash}
The Pedersen hash takes input strings of arbitrary length and outputs elements in the group $\gedcp$ of $\edcp$ - $\phash: \{0,1\}^* -> \gedcp$. Each group element is defined by $(x,y) \in \Fp^2$, where $p$ has a bit size of 377, and therefore a byte size of 48. We then define $\ulpserialize(x,y): \Fp^2 \rightarrow \{0,1\}^{392}$ as $\ulpserialize(x,y) := x$. Finally, we define $\ulphash: \{0,1\}^* \rightarrow \{0,1\}^{392}$ as $\ulpserialize \circ \phash $.

\paragraph{Hash to field}
To hash into the group $\G2$, where each element in defined by $(x,y) \in \Fptwo^2$, we first have to hash into the field $\Fptwo$. We invoke $\bhash$ multiple times to get enough random-looking bits to generate $x = (x_0, x_1)$ and take each inner element modulo $p$. We generate extra bits reduce modulo bias. Then, we see if $x$ is a valid $x$ on the curve, by trying to find a matching $y$.

Specifically, given a function we define $\ulfieldhash(m): \{0,1\}^* \rightarrow \Fp$ as follows:

First, we calculate 1024 random-looking bits using:
\begin{align*}
\bhash(\texttt{0x00000000} || m) &\ ||\ \bhash(\texttt{0x00000001} || m)\ ||\ \\
\bhash(\texttt{0x00000002} || m) &\ ||\ \bhash(\texttt{0x00000003} || m)
\end{align*}

We then divide it into two $512$ bits for $x_0$ and $x_1$, and for each we parse the bits as a little-endian integer and take it $\mod p$.

\paragraph{Hash to $\G2$}
We use the try-and-increment method - we initialize a counter, hash the counter together with the message as a possible $x$ value, attempt to find a matching $y$ and if we succeed, we multiply by the cofactor.

\begin{algorithm}
	\SetKwInOut{Input}{Input}
	\SetKwInOut{Output}{Output}
	\Input{$m$, a message}
	$hm := \ulphash(m)$ \\
	Initialize $c := 0$ \\
	Serialize $c$ as a little-endian 32-bit unsigned integer, and store in $cb$ \\
	$x := \ulfieldhash(c || hm)$ \\
	Attempt finding a matching $y$ by calculating $\sqrt{x^2 + 1}$. If not successful, increment $c$ and go back to step $2$. \\
	If successful, take the larger $y$, and return $\mathtt{cofactor} \cdot (x, y)$.
	.
	
	\caption{Try-and-increment hashing to the group}
\end{algorithm}


\bibliographystyle{alpha}
\bibliography{References}

\end{document}
 
